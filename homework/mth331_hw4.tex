\documentclass[11pt]{article}
\usepackage[margin=1in]{geometry}
\usepackage{amsmath, amssymb}

\title{MTH 331 -- Homework 4 Final Draft}
\author{Joe White}
\date{\today}

\begin{document}
\maketitle

% =========================================================
\section*{Exercise 1}

Assume, towards a contradiction, that $\sqrt{2}$ is rational. Then it can be written
\[
\sqrt{2} = \frac{a}{b}
\]
for some integers $a,b>0$. We are not assuming $\tfrac{a}{b}$ is in lowest terms.

Squaring both sides gives
\[
2 = \frac{a^2}{b^2} \quad \Rightarrow \quad a^2 = 2b^2.
\]
So $a^2$ is even, which means $a$ must be even. This is because if $a$ were odd, then $a^2$ would also be odd.  
Hence $a=2k$ for some integer $k$. Substituting back,
\[
(2k)^2 = 2b^2 \quad \Rightarrow \quad 4k^2 = 2b^2 \quad \Rightarrow \quad b^2 = 2k^2.
\]
Thus $b^2$ is even, so $b$ is even as well. In other words, both $a$ and $b$ are divisible by $2$.

Define $a_1 = a/2$ and $b_1 = b/2$. Then
\[
\sqrt{2} = \frac{a}{b} = \frac{a_1}{b_1}.
\]
Applying the same reasoning again, we find that $a_1$ and $b_1$ are also even.  
We can therefore define $a_2 = a_1/2$ and $b_2 = b_1/2$, and in general,
\[
a_{n+1} = \frac{a_n}{2}, \qquad b_{n+1} = \frac{b_n}{2}.
\]
This process repeats indefinitely, producing an infinite sequence $a_0, a_1, a_2, \dots$ of positive integers, each smaller than the one before it.

By the well-ordering principle, the set $\{a_0, a_1, a_2, \dots\}$ must have a least element.  
However, it cannot, since each $a_{n+1}$ is less than $a_n$.  
This contradiction shows that our original assumption was false.

Therefore, $\sqrt{2}$ is irrational.

\bigskip

% =========================================================
\section*{Exercise 2: Logical equivalences and examples}

We want to show
\[
(P \wedge Q \Rightarrow R) \;\equiv\; (P \wedge \sim R \Rightarrow \sim Q).
\]

Starting with $P \wedge Q \Rightarrow R$, we can use the equivalence $A \Rightarrow B \equiv (\sim A \vee B)$:
\[
P \wedge Q \Rightarrow R \;\equiv\; \sim(P \wedge Q) \vee R.
\]
This simplifies to
\[
(\sim P \vee \sim Q) \vee R.
\]

Now start with the second statement and apply the same steps:
\[
(P \wedge \sim R \Rightarrow \sim Q) \;\equiv\; \sim(P \wedge \sim R) \vee \sim Q.
\]
This becomes
\[
(\sim P \vee R) \vee \sim Q.
\]
Since both expressions simplify to a disjunction of the same three parts $(\sim P, \sim Q, R)$ in some order, they are logically equivalent.

\bigskip
\noindent This logical equivalence can be used to prove the following statements.

\medskip
\noindent\textbf{(a) Nonzero rational $\times$ irrational = irrational.}  
Let $r$ be a nonzero rational and $s$ be irrational.  
Suppose, for contradiction, that $rs$ is rational.  
Then $s = (rs)/r$. Since a rational number divided by a nonzero rational number is rational, this would make $s$ rational — a contradiction.  
Therefore the product must be irrational.

\medskip
\noindent\textbf{(b) Nonzero integer $\times$ noninteger = noninteger.}  
This statement is false.  
Example: $m=2$, $x=\tfrac{1}{2}$. Then $m$ is an integer, $x$ is noninteger, but $mx=1$ is an integer.  
The step that fails here is the assumption that multiplying a noninteger by a nonzero integer always produces another noninteger.  
This shows the conclusion of the logical equivalence does not hold for this case.

\medskip
\noindent\textbf{(c) Integer $+$ noninteger = noninteger.}  
If $m$ is an integer and $x$ is noninteger, assume for contradiction that $m+x$ is an integer.  
Then $x = (m+x) - m$ would also be an integer, contradicting the assumption that $x$ is noninteger.  
So the sum of an integer and a noninteger is always noninteger.  
For example, $1 + \tfrac{1}{2} = \tfrac{3}{2}$.

\bigskip

(a) is true, (b) is false (counterexample $2 \cdot \tfrac{1}{2} = 1$), and (c) is true.

\end{document}
