\documentclass[11pt]{article}
\usepackage[margin=1in]{geometry}
\usepackage{amsmath, amssymb}

\title{MTH 331 - Homework 4 Draft}
\author{Joe White}
\date{\today}

\begin{document}
\maketitle

% =========================================================
\section*{Exercise 1:}

Assume towards a contradiction, that $\sqrt{2}$ is rational. Then it can be written
\[
\sqrt{2} = \frac{a}{b}
\]
for some integers $a, b > 0$. We are not assuming $\tfrac{a}{b}$ is in lowest terms.

Squaring both sides gives
\[
2 = \frac{a^2}{b^2} \quad \Rightarrow \quad a^2 = 2b^2.
\]
So $a^2$ is even, which means $a$ must be even. $a = 2k$ for some integer $k$. Substituting back,
\[
(2k)^2 = 2b^2 \quad \Rightarrow \quad 4k^2 = 2b^2 \quad \Rightarrow \quad b^2 = 2k^2.
\]
Thus $b^2$ is even, so $b$ is even as well. In other words, both $a$ and $b$ are divisible by $2$.  

Define $a_1 = a/2$ and $b_1 = b/2$. Then
\[
\sqrt{2} = \frac{a}{b} = \frac{a_1}{b_1}.
\]
The exact same reasoning shows $a_1, b_1$ are both even, and so on for $a_2, b_2, \dots$  

This leads to an infinite descent. That cannot happen, since by the well-ordering principle, the natural numbers must eventually reach a smallest element.  

Therefore our assumption was wrong, and $\sqrt{2}$ is irrational.

\bigskip

% =========================================================
\section*{Exercise 2: Logical equivalences and examples}

We want to show
\[
(P \wedge Q \Rightarrow R) \;\equiv\; (P \wedge \sim R \Rightarrow \sim Q).
\]

Starting with $P \wedge Q \Rightarrow R$, we rewrite it as
\[
\sim (P \wedge Q) \vee R.
\]
This expands to
\[
(\sim P \vee \sim Q) \vee R.
\]
This is the same as: if $P$ is true and $R$ is false, then $Q$ must also be false. Which is equivalent to
\[
(P \wedge \sim R) \Rightarrow \sim Q.
\]
So the equivalence is proved.

\bigskip

\noindent \textbf{(a) Nonzero rational $\times$ irrational = irrational.}  
Suppose $r$ is a nonzero rational and $s$ is irrational. If $rs$ were rational, then $s=(rs)/r$ would also be rational, which is a contradiction. So the product must be irrational.

\medskip
\noindent \textbf{(b) Nonzero integer $\times$ noninteger = noninteger.}  
This statement is false. Example: $m=2$, $x=\tfrac{1}{2}$. Then $m$ is an integer, $x$ is noninteger, but $mx = 1$ is an integer. Dividing an integer by another integer does not always give an integer.

\medskip
\noindent \textbf{(c) Integer $+$ noninteger = noninteger.}  
If $m$ is an integer and $m+x$ were also an integer, then $x=(m+x)-m$ would be an integer, which contradicts the assumption that $x$ is noninteger. Therefore the sum of an integer and a noninteger is always noninteger. For instance, $1+ \tfrac{1}{2} = \tfrac{3}{2}$.

\bigskip
  
(a) is true,  
(b) is false (counterexample $2 \cdot \tfrac{1}{2} = 1$), 
(c) is true.  

\end{document}
