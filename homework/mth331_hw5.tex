\documentclass[11pt]{article}
\usepackage[margin=1in]{geometry}
\usepackage{amsmath, amssymb}

\title{MTH 331 - Homework \#5}
\author{Joe White}
\date{\today}

\begin{document}
\maketitle

% =========================================================
\section*{Exercise 2: Euclid’s Lemma and Prime Factorization}

\subsection*{Euclid’s Lemma for Finite Products}

\textbf{Statement.}  
If $p$ is prime and $p$ divides a product of positive integers,
\[
p \mid a_1 a_2 \cdots a_n,
\]
then $p$ divides at least one of the $a_i$.

\textbf{Proof.}  
We prove this by induction on $n$.

\emph{Base case ($n=2$):}  
This is the usual Euclid’s Lemma: if $p \mid a_1 a_2$, then $p \mid a_1$ or $p \mid a_2$.

\emph{Inductive step:}  
Assume the result holds for a product of $n$ factors.  
Now suppose $p \mid a_1 a_2 \cdots a_{n+1}$.  
Let $b = a_1 a_2 \cdots a_n$, so that $p \mid b a_{n+1}$.  
By the base case, either $p \mid b$ or $p \mid a_{n+1}$.  
If $p \mid a_{n+1}$, we are done.  
If $p \mid b$, then by the inductive hypothesis $p$ divides one of $a_1, a_2, \ldots, a_n$.  
Thus $p$ divides at least one of the $a_i$’s for $i = 1, 2, \ldots, $ $n+1$.  

Therefore, by induction, the statement holds for all $n \in \mathbb{Z}^+$.  
\hfill $\square$

\bigskip
\subsection*{Existence of Prime Factorizations}

\textbf{Statement.}  
Every integer $n > 1$ can be written as a product of primes.

\textbf{Proof (by least counterexample).}  
Suppose not, and let $n_0 > 1$ be the smallest integer that cannot be written as a product of primes.  

If $n_0$ is prime, then it already has a prime factorization, a contradiction.  
If $n_0$ is composite, then $n_0 = ab$ for some integers $a,b$ with $1 < a,b < n_0$.  
By minimality of $n_0$, both $a$ and $b$ can be written as products of primes:
\[
a = p_1 p_2 \cdots p_r, \qquad b = q_1 q_2 \cdots q_s.
\]
Multiplying gives
\[
n_0 = ab = p_1 p_2 \cdots p_r \, q_1 q_2 \cdots q_s,
\]
which is also a product of primes, again a contradiction.  
Hence, every integer $n > 1$ can be expressed as a product of primes.  

\textbf{Remark.}  
This result cannot be proved naturally by induction on $n$.  
A proof by induction would require a relationship between the prime factorization of $n_0$ and that of $n_0 - 1$, since the inductive hypothesis would only apply to $n_0 - 1$.  
However, consecutive integers do not have any simple relationship between their prime factorizations.  
In the least-counterexample proof, when $n_0$ is composite, we instead rely on two smaller numbers $a,b < n_0$ with $n_0 = ab$.  
Because the argument depends on such smaller factors rather than the immediately preceding integer, the inductive hypothesis would not be sufficient, and a least-counterexample argument is more appropriate here.

% =========================================================
\bigskip
\section*{Exercise 1: Proofs by Least Counterexample}

\subsection*{Sum of the First $n$ Odd Numbers}
\textbf{Claim.} For all $n \in \mathbb{Z}^+$,
\[
1 + 3 + 5 + \cdots + (2n-1) = n^2.
\]

\textbf{Proof (least counterexample).}
Assume the set
\[
S = \left\{ n \in \mathbb{Z}^+ \;:\; 1+3+\cdots+(2n-1) \neq n^2 \right\}
\]
is nonempty, and let $n_0$ be its least element.  
Then the statement is true for every $1 \le k < n_0$, and in particular
\[
1+3+\cdots+\bigl(2(n_0-1)-1\bigr) = (n_0-1)^2.
\]
Add the next odd term $2n_0-1$ to both sides:
\[
1+3+\cdots+\bigl(2(n_0-1)-1\bigr) + (2n_0-1) 
= (n_0-1)^2 + (2n_0-1) = n_0^2.
\]
But the left-hand side is exactly $1+3+\cdots+(2n_0-1)$, so the equality holds at $n_0$, contradicting the definition of $n_0$.  
Therefore $S$ is empty and the formula holds for all $n \in \mathbb{Z}^+$. \hfill $\square$

\subsection*{Sum of the First $n$ Integers}
\textbf{Claim.} For all $n \in \mathbb{Z}^+$,
\[
1 + 2 + 3 + \cdots + n \;=\; \frac{n(n+1)}{2}.
\]

\textbf{Proof (least counterexample).}
Assume the set
\[
T = \left\{ n \in \mathbb{Z}^+ \;:\; 1+2+\cdots+n \neq \frac{n(n+1)}{2} \right\}
\]
is nonempty, and let $n_0$ be its least element.  
Then for every $1 \le k < n_0$,
\[
1+2+\cdots+k \;=\; \frac{k(k+1)}{2}.
\]
Applying this at $k = n_0-1$ and then adding $n_0$ to both sides gives
\[
1+2+\cdots+(n_0-1)+n_0 
= \frac{(n_0-1)n_0}{2} + n_0
= \frac{n_0(n_0-1+2)}{2}
= \frac{n_0(n_0+1)}{2}.
\]
Hence the identity also holds at $n_0$, contradicting the choice of $n_0$.  
Therefore $T$ is empty and the formula holds for all $n \in \mathbb{Z}^+$. \hfill $\square$

\subsection*{Which Method Feels More Natural?}
For these two summation formulas, induction feels more natural: each case at $n+1$ is obtained by adding the ``next term'' to the case at $n$.  
Least counterexample works (and is formally equivalent to induction over $\mathbb{Z}^+$), but it is especially well-suited when a minimal failure leads you to factor or otherwise use \emph{some} smaller numbers (not just $n-1$), as in the existence of prime factorizations.

\end{document}
