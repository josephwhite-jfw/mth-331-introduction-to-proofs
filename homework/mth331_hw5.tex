\documentclass[11pt]{article}
\usepackage[margin=1in]{geometry}
\usepackage{amsmath, amssymb}

\title{MTH 331 - Homework \#5}
\author{Joe White}
\date{\today}

\begin{document}
\maketitle

\section*{Exercise 2: Euclid’s Lemma and Prime Factorization}

\subsection*{Euclid’s Lemma for Finite Products}

\textbf{Statement.}  
If $p$ is prime and $p$ divides a product of positive integers,
\[
p \mid a_1 a_2 \cdots a_n,
\]
then $p$ divides at least one of the $a_i$.

\textbf{Proof.}
We prove this by induction on $n$.

\emph{Base case ($n=2$):}
This is the usual Euclid’s Lemma: if $p \mid a_1 a_2$, then $p \mid a_1$ or $p \mid a_2$.

\emph{Inductive step:}
Assume the result holds for a product of $n$ factors.
Now suppose $p \mid a_1 a_2 \cdots a_{n+1}$.
Let $b = a_1 a_2 \cdots a_n$, so that $p \mid b a_{n+1}$.
By the base case, either $p \mid b$ or $p \mid a_{n+1}$.
If $p \mid a_{n+1}$, we are done.
If $p \mid b$, then by the inductive hypothesis $p$ divides one of $a_1, a_2, \ldots, a_n$.
Thus $p$ divides at least one of the $a_i$ for $i = 1, \ldots, n+1$.

Therefore the statement holds for all $n \in \mathbb{Z}^+$.

\bigskip
\subsection*{Existence of Prime Factorizations}

\textbf{Statement.}
Every integer $n > 1$ can be written as a product of primes.

\textbf{Proof (by least counterexample).}
Suppose not, and let $n_0 > 1$ be the smallest integer that cannot be written as a product of primes.

If $n_0$ is prime, then it already has a prime factorization, a contradiction.
If $n_0$ is composite, then $n_0 = ab$ for some integers $a,b$ with $1 < a,b < n_0$.
By minimality of $n_0$, both $a$ and $b$ can be written as products of primes:
\[
a = p_1 p_2 \cdots p_r, \qquad b = q_1 q_2 \cdots q_s.
\]
Multiplying gives
\[
n_0 = ab = p_1 p_2 \cdots p_r\, q_1 q_2 \cdots q_s,
\]
which is again a product of primes, contradiction.  
Hence every integer $n > 1$ can be expressed as a product of primes.

\textbf{Remark.}
A proof by induction would require a relationship between the factorization of $n_0$ and that of $n_0 - 1$,
since the inductive hypothesis would only apply to $n_0 - 1$.
However, consecutive integers do not have any simple relationship between their prime factorizations.
In the argument above, when $n_0$ is composite, we instead use two arbitrary smaller integers $a,b < n_0$ with $n_0 = ab$.
Since such information does not follow from the case $n_0 - 1$, an inductive hypothesis would not be strong enough.
A least-counterexample argument is therefore more appropriate here.

% =========================================================
% =========================================================
\bigskip
\section*{Exercise 3:}

Assume all $a_i$ and $a$ are real numbers, and any bases
appearing in exponents are positive to avoid complex number.

\subsection*{Part I.}

\textbf{Claim.}
For all $n \in \mathbb{Z}^+$,
\[
b(a_1 + a_2 + \cdots + a_n)
= ba_1 + ba_2 + \cdots + ba_n.
\]

\textbf{Proof.}
This can be proved with induction.
\emph{Base case ($n=1$):}
When the sum has only one term, the expression becomes
\[
b(a_1) = ba_1,
\]
which is true by the ordinary distributive law.  
Thus the statement holds for $n=1$.

\emph{Inductive step:}
Assume that the distributive property holds for a sum of $n$ terms; that is,
\[
b(a_1 + \cdots + a_n) = ba_1 + \cdots + ba_n.
\]
We must show that it then holds for a sum of $n+1$ terms.  
Starting with the longer sum,
\[
b(a_1 + \cdots + a_n + a_{n+1}),
\]
we group the first $n$ terms together:
\[
b\bigl((a_1 + \cdots + a_n) + a_{n+1}\bigr).
\]
Using the distributive law,
\[
b(a_1 + \cdots + a_n) + ba_{n+1}.
\]
Now we apply the inductive hypothesis to the first part:
\[
(ba_1 + \cdots + ba_n) + ba_{n+1},
\]
which is exactly what we want for $n+1$ terms.

Therefore the statement holds for all $n \in \mathbb{Z}^+$ by induction.

\subsection*{Part II.}

\textbf{Claim.}
For all $n \in \mathbb{Z}^+$,
\[
(a_1 a_2 \cdots a_n)^b = a_1^b a_2^b \cdots a_n^b.
\]

\textbf{Proof.}
This can also be proved by induction.

\emph{Base case ($n=1$):}
When there is only one factor,
\[
(a_1)^b = a_1^b,
\]
which is true by definition.

\emph{Inductive step:}
Assume the identity holds for a product of $n$ terms:
\[
(a_1 \cdots a_n)^b = a_1^b \cdots a_n^b.
\]
Want to show that it then holds for $n+1$ terms.  
Consider the product with one additional factor:
\[
(a_1 a_2 \cdots a_n a_{n+1})^b.
\]
Group first $n$ terms together:
\[
\bigl((a_1 \cdots a_n)\, a_{n+1}\bigr)^b.
\]
Using the rule $(xy)^b = x^b y^b$ for positive $x,y$,
\[
(a_1 \cdots a_n)^b \, a_{n+1}^b.
\]
Finally, apply the inductive hypothesis to the first factor:
\[
(a_1^b \cdots a_n^b)\, a_{n+1}^b,
\]
which is the desired expression.

Thus, by induction, the identity holds for all $n \in \mathbb{Z}^+$.

\end{document}
