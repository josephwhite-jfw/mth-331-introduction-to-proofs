\documentclass[11pt]{article}

\usepackage[margin=1in]{geometry}
\usepackage{amsmath, amssymb}

\title{MTH 331 — Homework \#1 (Rough Draft)}
\author{Joe White}
\date{\today}

\begin{document}
\maketitle

\section*{Exercise 2: Finite and Countably Infinite Sets}

Are the sets $A$, $B$, $C$, and $Q^+$ are finite or countably infinite.

\subsection*{Set $A$}
The set is 
\[
A = \{0, 0.01, 0.02, 0.03, \dots, 1\}.
\]
This has only finitely many elements (from $0$ up to $1$ in increments of $0.01$). If you counted them all up, there are $101$ of them, so $A$ is finite.

\subsection*{Set $B$}
The set is
\[
B = \{0, 0.9, 0.99, 0.999, \dots, 1\}.
\]
This is different from Set $A$, but we can list its elements in a sequence:
\[
0,\,0.9,\,0.99,\,0.999,\,\dots
\]
This list will go on forever and every element is some natural number $n$. So $B$ is countably infinite.

\subsection*{Set $C$}
% --- Rough draft note: explain how to rewrite $C$ into a single sequence
The set includes the same type of decimals as in $B$, but repeated for $1, 2, \dots$ and so on.
It can still be arranged into a single infinite list, which shows that $C$ is countably infinite.

\subsection*{Set $Q^+$}
% --- Rough draft note: explain why the positive rationals can be listed
The positive rationals can also be arranged into a sequence (for example, by writing them in a grid and reading diagonally). Therefore $Q^+$ is countably infinite.

\end{document}
