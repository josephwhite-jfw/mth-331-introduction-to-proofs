\documentclass[11pt]{article}

\usepackage[margin=1in]{geometry}
\usepackage{amsmath, amssymb}
\usepackage{amsfonts}

\title{MTH 331 — Homework \#1 Final Draft}
\author{Joe White}
\date{\today}

\begin{document}
\maketitle

\section*{Exercise 2: Finite and Countably Infinite Sets}

\subsection*{Set $A$}
The set is 
\[
A = \{0, 0.01, 0.02, 0.03, \dots, 1\}.
\]
We can list these numbers as
\[
0,\,0.01,\,0.02,\,0.03,\,\dots,\,1.
\]
This list stops after $101$ steps, so we can match each element to a natural number from $0$ up to $100$. 
Since the matching ends, $A$ is finite.

\subsection*{Set $B$}
The set is
\[
B = \{0, 0.9, 0.99, 0.999, \dots, 1\}.
\]
If we try to list it in the order written, $1$ would appear after infinitely many terms. To fix this, we 
reorder the set so that $1$ comes early, for example:
\[
0,\,1,\,0.9,\,0.99,\,0.999,\,\dots
\]
Now every element of $B$ shows up at some finite position. This gives a clear matching between $B$ and the 
natural numbers, so $B$ is countably infinite.

\subsection*{Set $C$}
The set is
\[
C = \{0, 0.9, 0.99, 0.999, \dots, 1, 1.9, 1.99, 1.999, \dots\}.
\]
There are two chains of decimals, one starting with $0$ and one starting with $1$. We can reorder them 
to make a single sequence:
\[
0,\,1,\,0.9,\,1.9,\,0.99,\,1.99,\,0.999,\,1.999,\,\dots
\]
This way each element of $C$ has a finite position in the list, so $C$ matches with the natural numbers and 
is countably infinite.

\subsection*{Set $\mathbb{Q}^+$}
The positive rationals can be written as fractions $\tfrac{p}{q}$ with $p,q \in \mathbb{N}$. If we put them 
in a grid (row $p$, column $q$), we can reorder them into a single sequence by reading along diagonals:
\[
\frac{1}{1},\;\frac{1}{2},\,\frac{2}{1},\;\frac{1}{3},\,\frac{2}{2},\,\frac{3}{1},\;\frac{1}{4},\,\frac{2}{3},\,\frac{3}{2},\,\frac{4}{1},\;\dots
\]
If we skip duplicates like $\tfrac{2}{2}=1$, this gives a list where each rational appears at a finite 
position. That means $\mathbb{Q}^+$ can be matched with the natural numbers and is countably infinite.

\section*{Exercise 3: Zermelo vs. von Neumann naturals}

\paragraph{The definitions.}
In Zermelo’s version, we start with $0=\varnothing$, and each new number is the singleton of the previous one:
\[
1=\{0\}=\{\varnothing\}, \quad
2=\{1\}=\{\{\varnothing\}\}, \quad
3=\{2\}=\{\{\{\varnothing\}\}\}, \quad
4=\{3\}=\{\{\{\{\varnothing\}\}\}\}, \ldots
\]
So each number just wraps the one before it.

In von Neumann’s version, we also start with $0=\varnothing$, but each new number is the set of all smaller numbers:
\[
1=\{0\}=\{\varnothing\}, \quad
2=\{0,1\}, \quad
3=\{0,1,2\}, \quad
4=\{0,1,2,3\}, \ldots
\]
So here $n$ is literally the collection of all numbers less than $n$.

\paragraph{Membership and order.}
In von Neumann’s version, $m\in n$ if and only if $m<n$. For example, $2\in 3$ and also $2<3$. This works because $n$ is defined to be the set of all smaller numbers. In Zermelo’s version, this is not true. For instance, $0<2$ but $0\notin 2$ since $2=\{1\}$.

\paragraph{Subsets and inequality.}
In von Neumann’s definition, $m\subseteq n$ if and only if $m\le n$. For example, $\{0,1\}\subseteq\{0,1,2\}$, so $2\subseteq 3$ and indeed $2\le 3$. In Zermelo’s version this fails. As an example, $1\le 2$ but $1=\{\varnothing\}$ is not a subset of $2=\{1\}$.

\paragraph{Cardinality.}
In von Neumann’s version, the size of $n$ is exactly $n$. For instance, $3=\{0,1,2\}$ and has three elements. In Zermelo’s version, every number greater than $0$ has only one element. For example, $3=\{2\}$, so $|3|=1$.

\paragraph{Why these differences occur.}
The key difference is that von Neumann’s construction keeps all the earlier numbers inside each new one, while Zermelo’s only keeps the immediate predecessor. This is why the von Neumann naturals line up so well with ordinary arithmetic and order properties.

\paragraph{Philosophical analogy.}
The repeating thought “I think”, “I think that I think”, “I think that I think that I think”, and so on is similar to Zermelo’s version, since each new stage is just one more layer around the last thought.

\paragraph{Which is more useful.}
In practice, von Neumann’s version is more convenient. It automatically makes properties like $m\in n$ corresponding to $m<n$, and it gives sets with the right size. Because of this, I would say that it is more useful

\end{document}
