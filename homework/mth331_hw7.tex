\documentclass[11pt]{article}
\usepackage[margin=1in]{geometry}
\usepackage{amsmath, amssymb}

\title{MTH 331 -- Homework 7}
\author{Joe White}
\date{\today}

\begin{document}
\maketitle

% =========================================================
\section*{Exercise 1}

Consider the relation $R$ on $\mathbb{Q}$ defined by
\[
R = \{(x,y) \in \mathbb{Q} \times \mathbb{Q} : x - y \in \mathbb{Z}\}.
\]
In other words, $xRy$ means that $x$ and $y$ differ by an integer.

\subsection*{(a) $R$ is an equivalence relation}

\textbf{Reflexive.}
Take any $x \in \mathbb{Q}$. Then
\[
x - x = 0,
\]
and $0$ is an integer. So $(x,x) \in R$, and $R$ is reflexive.

\medskip
\textbf{Symmetric.}
Now suppose $(x,y) \in R$. Then $x - y$ is an integer; write $k = x - y$. Then
\[
y - x = -(x - y) = -k.
\]
The integers are closed under taking negatives, so $-k \in \mathbb{Z}$, which means $y - x \in \mathbb{Z}$ and hence $(y,x) \in R$. So $R$ is symmetric.

\medskip
\textbf{Transitive.}
Finally, suppose $(x,y) \in R$ and $(y,z) \in R$. Then $x - y$ and $y - z$ are both integers. Let $m = x - y$ and $n = y - z$. Then
\[
x - z = (x - y) + (y - z) = m + n.
\]
Since the integers are closed under addition, $m + n$ is an integer. So $x - z \in \mathbb{Z}$ and $(x,z) \in R$. This shows $R$ is transitive.

\medskip
$R$ is reflexive, symmetric, and transitive, so $R$ is an equivalence relation on $\mathbb{Q}$.

\subsection*{(b) Infinitely many equivalence classes}

For $x \in \mathbb{Q}$, the equivalence class of $x$ is
\[
[x] = \{ y \in \mathbb{Q} : x - y \in \mathbb{Z} \}.
\]
So $[x]$ consists of all rationals that differ from $x$ by an integer.

To see that there are infinitely many different classes, it is convenient to pick one representative from the interval $[0,1)$ for each class. Let $r,s \in \mathbb{Q} \cap [0,1)$ with $r \neq s$. Then $r - s$ is a rational number with
\[
-1 < r - s < 1.
\]
The only integer between $-1$ and $1$ is $0$. Since $r \neq s$, we have $r - s \neq 0$, so $r - s \notin \mathbb{Z}$. That means $r$ and $s$ are not related, so $[r] \neq [s]$.

There are infinitely many rational numbers in $[0,1)$, and each one gives a distinct equivalence class. Therefore $R$ has infinitely many equivalence classes.

\subsection*{(c) Examples}

Here are examples (or non-examples) for each type of partition.

\medskip
\textbf{I. A partition of $\mathbb{Z}$ (or $\mathbb{N}$) into finitely many infinite sets.}

Fix an integer $k \ge 2$ and look at congruence classes mod $k$. On $\mathbb{Z}$, define
\[
A_j = \{ n \in \mathbb{Z} : n \equiv j \pmod{k} \},
\quad j = 0,1,\dots,k-1.
\]
Each $A_j$ is infinite, the sets $A_j$ are pairwise disjoint, and together they cover all of $\mathbb{Z}$. So $\{A_0,\dots,A_{k-1}\}$ is a partition of $\mathbb{Z}$ into finitely many infinite sets.

\medskip
\textbf{II. A partition of $\mathbb{Z}$ (or $\mathbb{N}$) into infinitely many finite sets.}

An easy example is the partition of $\mathbb{Z}$ into singletons:
\[
P = \{ \{n\} : n \in \mathbb{Z} \}.
\]
Each set $\{n\}$ has one element, so it is finite. The sets are pairwise disjoint and their union is all of $\mathbb{Z}$. There are infinitely many of them, so this gives a partition into infinitely many finite sets. (We could also group elements into pairs, e.g.\ $\{2n, 2n+1\}$, and get a similar example.)

\medskip
\textbf{III. A partition of $\mathbb{Z}$ (or $\mathbb{N}$) into finitely many finite sets.}

This cannot happen. Suppose, for contradiction, that
\[
\mathbb{Z} = B_1 \cup \cdots \cup B_m
\]
for some finite collection of finite sets $B_1,\dots,B_m$. A finite union of finite sets is finite, so the right-hand side would be finite. But $\mathbb{Z}$ is infinite, so this is impossible. The same argument shows there is no such partition of $\mathbb{N}$ either.

\medskip
\textbf{IV. A partition of $\mathbb{N}$ (or $\mathbb{Z}$) into infinitely many infinite sets.}

Here is a standard example on $\mathbb{N} = \{1,2,3,\dots\}$. Every positive integer can be written uniquely as
\[
n = 2^k m,
\]
where $k \ge 0$ and $m$ is odd. For each $k \ge 0$ define
\[
S_k = \{ n \in \mathbb{N} : n = 2^k m \text{ for some odd } m \}.
\]
Each $S_k$ is infinite (there are infinitely many odd numbers $m$), the sets $S_k$ are pairwise disjoint (the exponent $k$ of $2$ in this factorization is unique), and their union is all of $\mathbb{N}$. So $\{S_0, S_1, S_2, \dots\}$ is a partition of $\mathbb{N}$ into infinitely many infinite sets.

% =========================================================
\newpage
\section*{Exercise 2}

Recall that if $R$ is an equivalence relation on a set $A$, then the set of its equivalence classes
\[
P_R = \{ [a] : a \in A \}
\]
is a partition of $A$. Conversely, if $P$ is a partition of $A$, we can define a relation $R_P$ on $A$ by
\[
R_P = \bigcup_{B \in P} (B \times B).
\]

\subsection*{(a) The relation $R_P$ for a specific partition}

Let $A = \{0,1,2,3,4,5\}$ and
\[
P = \big\{ \{0\}, \{1,2\}, \{3,4,5\} \big\}.
\]
By definition,
\[
R_P = \bigcup_{B \in P} (B \times B).
\]
We can write this out block by block:
\[
\{0\} \times \{0\} = \{(0,0)\},
\]
\[
\{1,2\} \times \{1,2\} = \{(1,1), (1,2), (2,1), (2,2)\},
\]
\[
\{3,4,5\} \times \{3,4,5\}
= \{(3,3), (3,4), (3,5), (4,3), (4,4), (4,5), (5,3), (5,4), (5,5)\}.
\]
Putting these together,
\[
\begin{aligned}
R_P = {} & \{(0,0)\} \\
& \cup \{(1,1), (1,2), (2,1), (2,2)\} \\
& \cup \{(3,3), (3,4), (3,5), (4,3), (4,4), (4,5), (5,3), (5,4), (5,5)\}.
\end{aligned}
\]

So two elements of $A$ are related by $R_P$ exactly when they lie in the same part of the partition $P$. For instance, $1$ and $2$ are related, and $3$ and $5$ are related, but $0$ and $1$ are not, and $2$ and $3$ are not.

\subsection*{(b) In general, $R_P$ is an equivalence relation}

Now let $A$ be any set and let $P$ be a partition of $A$. This means:

\begin{itemize}
  \item Each $B \in P$ is a nonempty subset of $A$,
  \item The sets in $P$ are pairwise disjoint, and
  \item Every element of $A$ lies in at least one $B \in P$.
\end{itemize}

Define
\[
R_P = \bigcup_{B \in P} (B \times B).
\]

We check the three properties.

\medskip
\textbf{Reflexive.}
Let $a \in A$. Since $P$ is a partition, $a$ belongs to some block $B \in P$. Then $(a,a) \in B \times B$, so $(a,a) \in R_P$.

\medskip
\textbf{Symmetric.}
Suppose $(a,b) \in R_P$. Then $(a,b) \in B \times B$ for some $B \in P$, which means $a$ and $b$ are both in $B$. It follows that $(b,a) \in B \times B$ as well, so $(b,a) \in R_P$.

\medskip
\textbf{Transitive.}
Suppose $(a,b) \in R_P$ and $(b,c) \in R_P$. Then there are blocks $B_1, B_2 \in P$ such that
\[
(a,b) \in B_1 \times B_1
\quad \text{and} \quad
(b,c) \in B_2 \times B_2.
\]
So $a,b \in B_1$ and $b,c \in B_2$. In particular, $b \in B_1 \cap B_2$. But different blocks in a partition are disjoint, so if $B_1 \cap B_2$ is nonempty, we must have $B_1 = B_2$. Call this common block $B$. Then $a,c \in B$, which means $(a,c) \in B \times B \subseteq R_P$.

\medskip
Therefore $R_P$ is reflexive, symmetric, and transitive, so it is an equivalence relation on $A$.

\end{document}