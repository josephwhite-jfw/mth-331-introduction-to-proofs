\documentclass{article}
\usepackage{amsfonts}
\usepackage{amsmath}
\usepackage{enumerate}
\usepackage{graphicx}
\usepackage{fullpage}
\usepackage{url}
\usepackage{hyperref}

\title{MTH 331 Homework 1}
\author{}
\date{}

\begin{document}

\maketitle

\noindent
Choose two (2) of the following exercises to write responses to in \LaTeX.  Your writing must be your own.\footnote{If any of your writing is taken from or based on a source other than this document, the textbook/assigned readings, or the lectures, you must cite that source.  Changing someone else's words does not eliminate the need to cite them as a source.  Citations should provide enough detail for me to find and read the source.  Using text generated by AI would be treated similarly to using text generated by another person who did the assignment for you (or any significant part of it.)  Namely, it would be considered plagiarism if you didn't cite that person or AI as the source, and if you did cite the source, I would request a copy of what they provided to you and only give you credit for improvements that you made to it---which might not be a lot if it was half-decent to begin with, since not every change counts as an improvement.  Assume that you will be graded on the product of your own mind and complete the assignment accordingly.}

Before the beginning of class on Friday, September 5, you should write a rough draft of a response to at least one (1) of the exercises,\footnote{In following weeks, you will usually be asked to submit a rough draft for each exercise that you intend to submit a final draft for.  But this week, part of Friday's assignment is learning to use \LaTeX, so I don't want to make it too burdensome.} and submit both the \texttt{.tex} file containing your response and the corresponding \texttt{.pdf} file generated from it under the assignment ``HW 1 draft" on Canvas.

You should be prepared to present your response to that exercise on Friday, September 5 in class, to one or more classmates and (if selected) to the entire class.

By 11:59 p.m. on Monday, September 8, you should submit a \texttt{.tex} file and the corresponding \texttt{.pdf} file generated from it, both under the assignment ``HW 1 final" on Canvas.  These files should contain your final draft of responses to two (2) of the exercises, one of them being one you submitted a rough draft of, and your revision should incorporate feedback that you received from your classmates and/or me.

A couple of convenient ways to get started with \LaTeX\ are:
\begin{itemize}
\item Create a free Overleaf account at \url{https://overleaf.com},\footnote{You can turn off the AI features in Overleaf by clicking the person icon in the bottom left for ``Account", clicking ``Account settings", scrolling down to ``AI features", and clicking ``Disable AI features."  I think they are more annoying than useful; they seem to be overly picky and their results seem to be often wrong or at least debatable.  In any case, you shouldn't be using them to rewrite any substantial amount of text because you should be learning how to do that in your own voice.} or

\item Download MikTeX, a free \LaTeX distribution for Windows, Mac, or Linux at \url{https://miktex.org/download}.
\end{itemize}




If you have any questions, ask me at \texttt{twilson@miamioh.edu}, in class, or in office hours.

\newpage

\section*{Exercise 1}

Consider the following argument that set-builder notation doesn't always succeed in building a set.

\begin{enumerate}
    \item Assume that for every property $P$ there is a set $\{x \mid P(x)\}$.  This assumption is called ``unrestricted comprehension." (Why is it called ``comprehension" and why is it called ``unrestricted''?)

    \item Taking $P(x)$ to be the property ``$x$ is a set", unrestricted comprehension implies that there is a set $\{x \mid x \text{ is a set}\}$.  Call it $U$.  (Why did I call it $U$?)

    \item Taking $P(x)$ to be the property ``$x \in A \text{ and } Q(x)$" instead, unrestricted comprehension implies the \emph{Subset axiom} (Why is it called this?): if $A$ is a given set and $Q$ is a given property, there is a set
    \[B = \{x \mid x \in A \text{ and }Q(x)\} = \{x \in A \mid Q(x)\}.\]
    
    \item Taking $A = U$ and taking $Q$ to be $x \notin x$, the Subset axiom says there is a set (Why is it called $R$?)
    \[R = \{x \mid x \text{ is a set and } x \notin x\} = \{x \in U \mid x \notin x\}.\]

    \item Then for every set $x$, we have $x \in R$ if and only if $x \notin x$. (Why? Also, $x \in x$ is a weird-looking property.  Show that under our assumptions it may be true or false: consider $U$ and $\emptyset$.)

    \item In particular, taking $x$ to be the set $R$ itself, we have $R \in R$ if and only if $R \notin R$.
    
    \item This is a contradiction (Why?) so our assumption of unrestricted comprehension must be false.
\end{enumerate}
The argument is mathematically correct but very dry.  \textbf{Rewrite} it to do the following:
\begin{itemize}
    \item Use paragraphs instead of bullet points.
    \item Use more words (relative to the amount of mathematical notation) to explain the concepts in a way that is more engaging and easier to read.  One way to do this is to answer some or all of the parenthetical questions and incorporate the answers smoothly into your text as explanations.
    \item Modern (Zermelo--Fraenkel) set theory assumes the Subset axiom, but not unrestricted comprehension.  In this case, what the argument shows is that there is no set of all sets.  \textbf{Rewrite/reorganize your writing to center it on the goal of showing there is no set of all sets, assuming that the subset axiom is true, but not assuming unrestricted comprehension}.
\end{itemize}

\newpage

\section*{Exercise 2}
 For a set $A$ to be \emph{finite} means that there is a way to write 
 \[A = \{a_0,a_1,a_2,\ldots, a_{n-1}\}\]
 where $n$ is a natural number called the cardinality of $A$, and $a_0,a_1,a_2,\ldots, a_{n-1}$ are distinct elements of $A$. (In this list there is an element $a_i$ for each natural number $i < n$.)\footnote{Note that in the case $n = 1$, the list of elements stops as soon as it starts and $A = \{a_0\}$.  In the case $n = 0$, the list stops \emph{before} it starts and we just have $A = \{\} = \emptyset$.}
 
 Otherwise, if the set cannot be written in this way for any natural number $n$, it is called \emph{infinite}.
 
 More specifically, for a set $A$ to be \emph{countably infinite} means that there is a way to write 
 \[A = \{a_0,a_1,a_2,\ldots\}\]
 where $a_0, a_1, a_2,\ldots$ are distinct elements of $A$.  (In this list there is an element $a_i$ for each natural number $i$.)

Consider the following sets:
\begin{align*}
A &= \{0,0.01, 0.02, 0.03, \ldots 1\}\\
B &= \{0,0.9, 0.99, 0.999, \ldots 1\}\\
C &= \{0,0.9, 0.99, 0.999, \ldots, 1,1.9, 1.99, 1.999, \ldots\}\\
\mathbb{Q}^+ &= \{\textstyle \frac{1}{1}, \frac{2}{1}, \frac{3}{1}, \frac{4}{1}, \frac{5}{1}, \frac{6}{1}, \ldots, \frac{1}{2}, \frac{3}{2}, \frac{5}{2}, \frac{7}{2}, \frac{9}{2}, \frac{11}{2}, \ldots, \frac{1}{3},\frac{2}{3}, \frac{4}{3}, \frac{5}{3}, \frac{7}{3}, \frac{8}{3},\ldots\}
\end{align*}
\textbf{Explain, in paragraphs using clear English and enough mathematical notation to make your writing precise, why $A$ is finite and $B$, $C$, and $\mathbb{Q}^+$ are countably infinite.}  For each of the sets $B$, $C$, and $\mathbb{Q}^+$, include an explanation of how/why it can/should be rewritten to show that it meets the definition of ``countably infinite", and how it differs from the previous set in the technique that needs to be used.

\newpage

\section*{Exercise 3}

When set theory is used as a basis for mathematics, all mathematical objects are encoded as sets.  For natural numbers, there are two main ways to do this.

Zermelo's definition:
\begin{align*}
    0 &= \emptyset\\
    1 &= \{0\} = \{\emptyset\}\\
    2 &= \{1\} = \{\{\emptyset\}\}\\
    3 & = \{2\} = \cdots\\
    \vdots
\end{align*}

Von Neumann's definition:
\begin{align*}
    0 &= \{\} = \emptyset\\
    1 &= \{0\} = \{\emptyset\}\\
    2 &= \{0,1\} = \{\emptyset,\{\emptyset\}\}\\
    3 & = \{0,1,2\} = \cdots\\
    \vdots
\end{align*}
\textbf{Explain, compare, and contrast these two definitions.}  Write your answer in paragraphs, addressing all the bullet points below, using clear English and sufficiently precise mathematical notation.
\begin{itemize}
    \item Write in words what each of these definitions says in general (i.e., what the pattern is) and extend the lists above to complete the definition of 3 and include the definition of 4.

    \item Which of the definitions has the property that $m \in n$ if and only if $m < n$?

    \item Which of the definitions has the property that $m \subseteq n$ if and only if $m \le n$?

    \item Which of the definitions has the property that $|n| = n$?

    \item In each case, explain in general why the one definition has the property, and give a specific counterexample to show that the other one doesn't.

    \item Which definition is analogous to the following philosophical argument that infinitely many different thoughts can be created from nothing: "I think", "I think that I think", "I think that I think that I think", etc.  Explain the analogy.

    \item Which definition seems to have mathematically more useful properties?  Why?
\end{itemize}

\end{document}
