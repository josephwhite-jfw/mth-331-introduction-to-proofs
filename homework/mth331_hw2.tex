\documentclass[11pt]{article}
\usepackage[margin=1in]{geometry}
\usepackage{amsmath, amssymb}

\title{MTH 331 — Homework \#2 (Rough Draft)}
\author{Joe White}
\date{\today}

\begin{document}
\maketitle

\section*{Exercise 2: Implication is not associative}

We want to compare the statements
\[
P \implies (Q \implies R) \quad \text{and} \quad (P \implies Q) \implies R.
\]

\subsection*{Truth tables}
Here is the beginning of the truth table for both statements. Each row corresponds to one choice of truth values for $P$, $Q$, and $R$.

\begin{center}
\begin{tabular}{|c|c|c||c|c|}
\hline
$P$ & $Q$ & $R$ & $P \implies (Q \implies R)$ & $(P \implies Q) \implies R$ \\
\hline
T & T & T & T & T \\
T & T & F & F & F \\
T & F & T & T & T \\
T & F & F & T & F \\
F & T & T & T & T \\
\hline
\end{tabular}
\end{center}

\subsection*{Explanation in words}
The statement $P \implies (Q \implies R)$ means that if $P$ is true, then we also need that $Q \implies R$ is true. This is the same as saying “if $P$ and $Q$ are both true, then $R$ must be true.” That is why $P \implies (Q \implies R)$ is logically equivalent to $(P \land Q) \implies R$.

An example in plain English: Let $P$ be “it is raining,” let $Q$ be “I have an umbrella,” and let $R$ be “I will stay dry.” The statement $P \implies (Q \implies R)$ means “if it is raining, then if I have an umbrella I will stay dry.” This is the same as saying “if it is raining and I have an umbrella, then I will stay dry.”

\end{document}
