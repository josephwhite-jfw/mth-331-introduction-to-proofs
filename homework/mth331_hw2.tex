\documentclass[11pt]{article}

\usepackage[margin=1in]{geometry}
\usepackage{amsmath, amssymb}
\usepackage{array}

\title{MTH 331 — Homework \#2}
\author{Joe White}
\date{\today}

\begin{document}
\maketitle

\section*{Exercise 2: Implication is not associative}

We can compare the statements
\[
P \Rightarrow (Q \Rightarrow R)
\quad\text{and}\quad
(P \Rightarrow Q) \Rightarrow R,
\]
and also relate them to $(P \land Q) \Rightarrow R$.

\subsection*{Full truth tables with intermediate columns}

\begin{center}
\renewcommand{\arraystretch}{1.15}
\begin{tabular}{|c|c|c||c|c||c|c|}
\hline
$P$ & $Q$ & $R$ & $Q\Rightarrow R$ & $P\Rightarrow(Q\Rightarrow R)$ & $P\Rightarrow Q$ & $(P\Rightarrow Q)\Rightarrow R$ \\
\hline
T & T & T & T & T & T & T \\
T & T & F & F & F & T & F \\
T & F & T & T & T & F & T \\
T & F & F & T & T & F & F \\
F & T & T & T & T & T & T \\
F & T & F & F & T & T & F \\
F & F & T & T & T & T & T \\
F & F & F & T & T & T & F \\
\hline
\end{tabular}
\end{center}

The last two columns differ on several rows, so
\[
P \Rightarrow (Q \Rightarrow R) \quad\text{is not logically equivalent to}\quad (P \Rightarrow Q) \Rightarrow R.
\]

\subsection*{Equivalence with $(P \land Q) \Rightarrow R$ by a truth table}

\begin{center}
\renewcommand{\arraystretch}{1.15}
\begin{tabular}{|c|c|c||c|c|}
\hline
$P$ & $Q$ & $R$ & $P \Rightarrow (Q \Rightarrow R)$ & $(P\land Q)\Rightarrow R$ \\
\hline
T & T & T & T & T \\
T & T & F & F & F \\
T & F & T & T & T \\
T & F & F & T & T \\
F & T & T & T & T \\
F & T & F & T & T \\
F & F & T & T & T \\
F & F & F & T & T \\
\hline
\end{tabular}
\end{center}

The two columns match in every row, so
\[
P \Rightarrow (Q \Rightarrow R) \quad\text{is logically equivalent to}\quad (P \land Q) \Rightarrow R.
\]

\subsection*{Why the equivalence is natural}

$P \Rightarrow (Q \Rightarrow R)$ says: if $P$ holds, then the inner implication $Q \Rightarrow R$ must hold. The inner implication fails only when $Q$ is true and $R$ is false. So the whole statement fails only in the case $P$ and $Q$ are both true and $R$ is false. That is exactly when $(P \land Q) \Rightarrow R$ fails. This is why they are equivalent.

\subsection*{Is $P \Rightarrow (Q \Rightarrow R)$ equivalent to $Q \Rightarrow (P \Rightarrow R)$?}

Yes. From above, $P \Rightarrow (Q \Rightarrow R)$ is equivalent to $(P \land Q) \Rightarrow R$. By the same reasoning with $P$ and $Q$ swapped, $Q \Rightarrow (P \Rightarrow R)$ is also equivalent to $(P \land Q) \Rightarrow R$. Since both are equivalent to the same statement, they are equivalent to each other.

\subsection*{Implication only form of $(P \land Q \land R) \Rightarrow S$}

Use the identity $A \land B \equiv \neg(A \Rightarrow \neg B)$. Applying it twice,
\[
(P \land Q \land R) \Rightarrow S
\;\;\equiv\;\;
\bigl(P \Rightarrow \neg(Q \Rightarrow \neg R)\bigr) \Rightarrow S,
\]
which uses only $\Rightarrow$ and $\neg$. It is correct because $Q \land R$ is true exactly when $Q \Rightarrow \neg R$ is false.

\bigskip

\section*{Exercise 3: The statement $((P \Rightarrow \neg P)\Rightarrow P)\Rightarrow P$ is a tautology}

Let
\[
S = ((P \Rightarrow \neg P)\Rightarrow P)\Rightarrow P.
\]

\subsection*{Truth table}

Since $S$ depends only on $P$, there are only two cases to check.

\begin{center}
\renewcommand{\arraystretch}{1.15}
\begin{tabular}{|c||c|c|c|}
\hline
$P$ & $P \Rightarrow \neg P$ & $(P \Rightarrow \neg P)\Rightarrow P$ & $S$ \\
\hline
T & F & T & T \\
F & T & F & T \\
\hline
\end{tabular}
\end{center}

In both rows $S$ is true, so the whole statement is a tautology.

\subsection*{Step-by-step equivalences}

We can also show $S$ is a tautology by simplifying it with standard equivalences.

\begin{align*}
S
&= ((P \Rightarrow \neg P)\Rightarrow P)\Rightarrow P \\
&\equiv \neg\bigl((P \Rightarrow \neg P)\Rightarrow P\bigr) \lor P
&& \text{rewrite outer implication} \\
&\equiv \bigl((P \Rightarrow \neg P) \land \neg P\bigr) \lor P
&& \text{negate inner implication} \\
&\equiv \bigl((\neg P \lor \neg P) \land \neg P\bigr) \lor P
&& \text{rewrite } P \Rightarrow \neg P \\
&\equiv (\neg P \land \neg P) \lor P
&& \text{idempotence of $\lor$} \\
&\equiv \neg P \lor P \\
&\equiv P \Rightarrow P.
\end{align*}

Since $P \Rightarrow P$ is always true, $S$ is a tautology.

\subsection*{Explanation in words}

Work from the inside out. The statement $P \Rightarrow \neg P$ is true only when $P$ is false, so it is equivalent to $\neg P$. That makes the middle piece $(P \Rightarrow \neg P)\Rightarrow P$ say “if $\neg P$ then $P$,” which is just another way of saying $P$. Finally $S$ becomes $P \Rightarrow P$, which is always true. This matches the everything above and confirms that $S$ is a tautology.

\end{document}
