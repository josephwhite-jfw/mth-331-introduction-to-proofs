\documentclass[11pt]{article}

\usepackage[margin=1in]{geometry}
\usepackage{amsmath, amssymb}
\usepackage{array}

\title{MTH 331 — Homework \#3 Draft}
\author{Joe White}
\date{\today}

\begin{document}
\maketitle

% =========================================================
\section*{Exercise 1: Slogans, logical form, and inference rules}

\paragraph{Orville Redenbacher slogan.}
Let $P$ mean “you will like it better.” Let $Q$ mean “my name is Orville Redenbacher.”  
The slogan
\[
\text{``You will like it better or my name is not Orville Redenbacher.''}
\]
has logical form
\[
P \lor \neg Q.
\]

If the slogan is true and the speaker’s name really is Orville Redenbacher, then $Q$ is true. From $Q$ and $P\lor\neg Q$ we can conclude $P$.

\medskip
\noindent\textbf{Inference rule.}
\[
\frac{P\lor\neg Q \quad\quad Q}{P}
\]

This works because $P\lor\neg Q$ is equivalent to $Q\Rightarrow P$. With $Q$ true, modus ponens lets us conclude $P$.

\medskip
\noindent\textbf{Truth table check.}  
In the rows where both premises are true, $P$ is also true.

\begin{center}
\renewcommand{\arraystretch}{1.15}
\begin{tabular}{|c|c||c|c||c|}
\hline
$P$ & $Q$ & $\neg Q$ & $P\lor\neg Q$ & Both premises true? $\Rightarrow P$ \\
\hline
T & T & F & T & Yes, $P$ is T \\
F & T & F & F & No \\
T & F & T & T & No \\
F & F & T & T & No \\
\hline
\end{tabular}
\end{center}

\paragraph{As a conditional.}
Since $\neg Q \lor P$ is equivalent to $Q \Rightarrow P$, the slogan can also be written as
\[
Q \Rightarrow P.
\]
With $Q$ true, modus ponens again gives $P$.

\paragraph{Bumper sticker.}
“If you’re not outraged, you’re not paying attention.”  
Let $P$ mean “you are paying attention.” Let $Q$ mean “you are outraged.”  
Logical form:
\[
\neg Q \Rightarrow \neg P.
\]
The contrapositive is $P \Rightarrow Q$, meaning if you are paying attention then you are outraged.

\paragraph{Why word it this way.}
The original versions sound sharper. The Orville slogan sounds confident, and the bumper sticker is provocative. The simpler conditionals are logically the same but the given wording is more attention-grabbing.

\bigskip

% =========================================================
\section*{Exercise 3: Parity statements with three integers}

Assume $m,n,r\in\mathbb{Z}$. An integer is even if it is $2k$ for some $k$, and odd if it is $2k+1$.

\subsection*{(1) If $m,n,r$ are all odd then $m+n+r$ is odd. \,\,\textit{True.}}
Write $m=2a+1$, $n=2b+1$, $r=2c+1$. Then
\[
m+n+r = (2a+1)+(2b+1)+(2c+1) = 2(a+b+c+1)+1,
\]
which is odd.

\subsection*{(2) If $m+n+r$ is odd then $m,n,r$ are all odd. \,\,\textit{False.}}
Example: $m=2$, $n=2$, $r=1$. Then $m+n+r=5$ is odd, but $m$ and $n$ are even.

\subsection*{(3) If at least one of $m,n,r$ is odd then $m+n+r$ is odd. \,\,\textit{False.}}
Example: $m=1$, $n=1$, $r=2$. At least one number is odd, but $m+n+r=4$ is even.  
Another example: $m=1$, $n=2$, $r=3$ gives $1+2+3=6$, which is even.

\subsection*{(4) If $m+n+r$ is odd then at least one of $m,n,r$ is odd. \,\,\textit{True.}}
Proof by contrapositive. Suppose none of $m,n,r$ is odd. Then all are even: $m=2a$, $n=2b$, $r=2c$. Then
\[
m+n+r = 2(a+b+c),
\]
which is even. So if the sum is odd, not all three can be even. That means at least one must be odd.

\paragraph{Why I used these methods.}
Parts (1) and (4) are easy to prove directly or directly with a contrapositive. Parts (2) and (3) are false, and counterexamples are the clearest way to show that.

\end{document}
